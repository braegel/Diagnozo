\documentclass{article}
\usepackage[T1]{fontenc}
\usepackage{lmodern}
\usepackage{graphicx}
\usepackage[utf8]{inputenc}
\usepackage{hyperref}
\usepackage{listings}

\title{Diagnozo}
\author{Bernd Brägelmann}
\date{\today}

\begin{document}
\maketitle
\tableofcontents

\begin{abstract}
Diagnozo will hopefully be a tool to allow finding a medical diagnosis based upon a tag based filtering system.
\end{abstract}

\section{The diagnosis database: Diagnozo.xml}

The \href{https://github.com/braegel/Diagnozo/blob/master/data/diagnozo.xml}{Diagnozo.xml}\footnote{Diagnozo is \href{http://en.wikipedia.org/wiki/Esperanto}{esperanto} for ``diagnosis''.} file contains the informations related to a diagnosis.

I suggest the following xml elements and attributes. Have a look at this example:

\lstset{language=xml}
\begin{lstlisting}[caption="First version of Diagnozo.xml",label="firstversionofdiagnozoxml",breaklines=true,frame=tlRB]
<diagnozo>
	<diagnosis name="Azygos lobe">
		<url>http://en.wikipedia.org/wiki/Azygos_lobe</url>
		<xray>small line in the upper lobe of the right lung<xray/>
	</diagnosis>
	<diagnosis name="Insufficient inspiration">
		<xray>unsharp dens hilar vessels<xray/>
		<xray>pulmonary increased streaking<xray>
		<differentialdiagnosis>Heart failure<differentialdiagnosis>
	</diagnosis>
</diagnozo>
\end{lstlisting}

\subsection{Attributes}

\begin{description}
	\item[name] name of the diagnosis
\end{description}

\subsection{Elements}
All elements can be used multiple times per diagnosis.

\begin{description}
	\item[url] urls
	\item[region] region of the body where the disease can occure
	\item[xray] how does the diagnosis look like in native x-ray images
	\item[ct] how does the diagnosis look like in computed tomography scans
	\item[mri] how does the diagnosis look like in computed tomography scans
\end{description}

\end{document}